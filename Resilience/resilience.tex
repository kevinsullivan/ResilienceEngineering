\documentclass[11pt]{article}
\begin{document}
\begin{titlepage}
  \begin{center}
    \line(1,0){300} \\
    [0.25in]
    \huge{\bfseries Resilience as an Engineered System Property} \\
    [2mm]
    \line(1,0){200} \\
    [1.5cm]
    \textsc{\LARGE Does it Make Sense? \\ What Sense does it Make?} \\
    [3cm]
    DRAFT WORKING PAPER \\
    [6.0cm]
  \end{center}
  \begin{flushright}
    Chong Tang, Kevin Sullivan, $\ldots$\\
  \end{flushright}
\end{titlepage}

%%%%%%%%%%%%%%%%%%%%%%%%%%%%%%%%%%%%%%%%%%%%%%%%%%%%%%%%%%%%%%%%%%%%%%%%

\begin{abstract}
  The concept of resilience as a system property has garnered
  considerable attention in recent years. Yet the status of the
  concept remains unsettled.  We do have various {\it mechanisms\/}
  that support diverse, often narrow conceptions of resilience. At the
  same time, the literature addressing this property is sparse; there
  is little in the way of a broadly shared, precise understanding of
  what constitutes such a property; it is unclear whether the putative
  property is redundant with other recognized system properties, such
  as survivability; and we lack notations for specifying resilience
  properties independently of underlying mechanisms, and methods for
  assuring the satisfaction of such resilience specifications.  This
  paper presents a survey and analysis of concepts of resilience as a
  system property, distinct from mechanisms, and assesses the status
  of the concept and needs for future research and development.
\end{abstract}

%%%%%%%%%%%%%%%%%%%%%%%%%%%%%%%%%%%%%%%%%%%%%%%%%%%%%%%%%%%%%%%%%%%%%%%%

\begin{center}
  {\Large \bf Bibliography}
\end{center}

\nocite{*}
\bibliographystyle{IEEEtran} \bibliography{resilience}

\end{document}
