\documentclass[11pt]{article}
\begin{document}
\begin{titlepage}
  \begin{center}
    \line(1,0){300} \\
    [0.25in]
    \huge{\bfseries Resilience as an Engineered System Property} \\
    [2mm]
    \line(1,0){200} \\
    [1.5cm]
    \textsc{\LARGE Does it Make Sense? \\ What Sense does it Make?} \\
    [3cm]
    DRAFT WORKING PAPER \\
    [6.0cm]
  \end{center}
  \begin{flushright}
    Chong Tang, Kevin Sullivan, $\ldots$\\
  \end{flushright}
\end{titlepage}

%%%%%%%%%%%%%%%%%%%%%%%%%%%%%%%%%%%%%%%%%%%%%%%%%%%%%%%%%%%%%%%%%%%%%%%%

\begin{abstract}
  The concept of resilience as a property of engineered systems has
  garnered considerable attention in recent years. Yet the status of
  the concept remains unsettled and our ability to specify, realize,
  and assure resilience properties remains weak.  We have conflicting
  conceptions of the origins of resilience, as either an engineered or
  organic property. The relationship of resilience to other recognized
  properties, such as survivability, is unclear. We have numerous but
  informal and inconsistent definitions of resilience. We have various
  {\it mechanisms\/} that support specific, often narrow notions of
  resilience. The resilience literature is sparse. We lack resilience
  specification languages and assurance methods.  This paper presents
  a survey and analysis of work on resilience as a system property,
  distinct from mechanisms, and assesses the status of the concept and
  needs for future research and development.
\end{abstract}

%%%%%%%%%%%%%%%%%%%%%%%%%%%%%%%%%%%%%%%%%%%%%%%%%%%%%%%%%%%%%%%%%%%%%%%%

\section{Definitions}

The International Council on Systems Engineering (INCOSE) defines resilience as ``the ability of organizational, hardware and software systems to mitigate the severity and likelihood of failures or losses, to adapt to changing conditions, and to respond appropriately after the fact~\cite{incose}.''

%%%%%%%%%%%%%%%%%%%%%%%%%%%%%%%%%%%%%%%%%%%%%%%%%%%%%%%%%%%%%%%%%%%%%%%%

\begin{center}
  {\Large \bf Bibliography}
\end{center}

\nocite{*}
\bibliographystyle{IEEEtran} \bibliography{resilience}

\end{document}
