\documentclass[11pt]{article}
\begin{document}
\begin{titlepage}
  \begin{center}
    \line(1,0){300} \\
    [0.25in]
    \huge{\bfseries Resilience as an Engineered System Property} \\
    [2mm]
    \line(1,0){200} \\
    [1.5cm]
    \textsc{\LARGE Does it Make Sense? \\ What Sense does it Make?} \\
    [3cm]
    DRAFT WORKING PAPER \\
    [6.0cm]
  \end{center}
  \begin{flushright}
    Chong Tang, Kevin Sullivan, $\ldots$\\
  \end{flushright}
\end{titlepage}

%%%%%%%%%%%%%%%%%%%%%%%%%%%%%%%%%%%%%%%%%%%%%%%%%%%%%%%%%%%%%%%%%%%%%%%%

\begin{abstract}
  The concept of resilience as a system property has garnered
  considerable attention in recent years. Yet the status of the
  concept remains unsettled.  The literature addressing this property
  is sparse. There is little in the way of a broadly shared, precise
  understanding of what constitutes such a property. It is unclear
  whether the putative property is redundant with other recognized
  properties, such as survivability. Moreover we lack both notations
  for specifying resilience properties and methods for assuring the
  satisfaction of such specifications. This paper presents a survey
  and analysis of concepts of resilience, distinguishing properties
  from mechanisms, and assesses the status of the concept and needs
  for future research.
\end{abstract}

%%%%%%%%%%%%%%%%%%%%%%%%%%%%%%%%%%%%%%%%%%%%%%%%%%%%%%%%%%%%%%%%%%%%%%%%

\begin{center}
  {\Large \bf Bibliography}
\end{center}

\nocite{*}
\bibliographystyle{IEEEtran} \bibliography{resilience}

\end{document}
